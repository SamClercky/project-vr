\documentclass[]{article}
\usepackage{vub}

\usepackage{amsmath}
\usepackage{amssymb}
\usepackage{amsthm}
\usepackage{minted}
\usepackage{graphicx}
\usepackage{subfig}
\usepackage[inkscapeformat=png]{svg}
\usepackage{parskip}

%\renewcommand{\includegraphics}{\relax}

\newcommand*{\TODO}{\textbf{TODO: }}
\newcommand*{\Gevolg}{\textbf{Gevolg: }}

\newcommand*{\so}{$\to$\ }
\newcommand*{\D}{\ensuremath{\mathcal{D}}}
\newcommand*{\T}{\ensuremath{\mathcal{T}}}
\newcommand*{\C}{\ensuremath{\mathbb{C}}}
\newcommand*{\R}{\ensuremath{\mathbb{R}}}
\newcommand*{\Q}{\ensuremath{\mathbb{Q}}}
\newcommand*{\Z}{\ensuremath{\mathbb{Z}}}
\newcommand*{\N}{\ensuremath{\mathbb{N}}}
\renewcommand*{\O}{\ensuremath{\mathcal{O}}}
\renewcommand*{\L}{\ensuremath{\mathcal{L}}}
\renewcommand*{\implies}{\ensuremath{\Longrightarrow}}
\renewcommand*{\inf}{\ensuremath{\infty}}

\renewcommand*{\boxed}[1]
{ \framebox{ \parbox[c]{0.9\textwidth}{
            #1
} } }

\title{Project VR}
\subtitle{Creating a game in OpenGL}
\author{Andreas Declerck, Sven Degelean}

\faculty{Faculty of Science and Bio-Engineering: Computer science}

\promotors{Academic year 2022-2023}
\date{\today}

\begin{document}
\maketitle

\section{Introduction}

This document describes the implemented features in the OpenGL project for the
course VR. The assigniment is to create a game in OpenGL with as many as possible
interesting features.

\TODO Insert image of finished game

\section{Description of the game}

The game places the player in an environment where interesting effects can be
observed. This while the player's vision gets distorted to make it more
interesting.

The player is constraint to physiscs like gravity and can interact with its
environment by colliding with the surrounding objects. A mechanic was added
to make it possible to shoot little balls from the direction of where the
player is looking. Player movement can be achieved with an XBox gamepad or with
wasd + mouse. Shooting works with pressing spacebar, while on the gamepad the LB
button is used.

Some debugging mechanics where also introduced to find more easily the
boundingboxes used for the physics. On the gamepad this is B or I on a keyboard.
Similarly a freecam option was introduced with button A on the gamepad and O on
a keyboard.

\TODO Insert image of gamepad + functions

\section{Technical features}

To make all of this work, certain special features where introduced. These
contain:
\begin{itemize}
    \item Input processing: gamepad or keyboard + mouse implemented with glfw
    \item Automatic resizing: gamescreen updates when its surrounding window
          changes size (glViewport + updated framebuffer textures).
    \item Skybox: to give a sense of orientation in the scene
    \item Particle effect (moving snow whirl): based on perlin noise texture and
          created with a geometry shader which becomes periodically transparent
          (glEnable(GL\_BLEND) + fragment shader).
    \item Collision detection: bullet3 + gravity
          \begin{itemize}
              \item Mode for viewing bounding boxes from bullet3 created with a
                    geometry shader
              \item The camera is bound to a collision box, so the player is
                    bound to the physics in the world.
              \item Freecam mode in which the player can detach itself from
                    collision detection boundaries.
          \end{itemize}
    \item Bullets: can be thrown and collide with other objects (fun to move
          things around with).
    \item Implemented light equations (phong shading): containing with
          attentuation and decreasing intensity with increasing distance:
          \begin{itemize}
              \item Ambient light
              \item Diffuse light
              \item Specular light
          \end{itemize}
    \item Directional shadows: for both point- and spotlights with
          framebuffers where the scene is rerendered for every light with the
          possibility to exclude certain objects that do not cast a shadow.
    \item Framebuffers postprocessing effect: periodically applies an edge
          detecting kernel function and adds a cross to help with aiming and uses
          inverted colors of the current frame for better visibilty of the cross.
    \item Ability to efficiently load and use external resources: by storing the
          path together with the requested resource, so duplicate calls return
          the same shared reference to the result.
    \item Ephimeral objects: Bullets can become to much in a scene and degrade
          the performance, so a system was created to remove objects if they
          fall out of the world (y < -100). Additionally a system of ephimeral
          objects was created.  These objects have a time to live (ttl) after
          which, they are destroyed.
\end{itemize}

\section{Organisation of the code and external resources}

This project would not have been possible without following tools and libraries:
\begin{itemize}
    \item GLAD:\ for easy OpenGL access
    \item GLFW:\ for cross-platform access to the windowing system
    \item GLM:\ for handy linear algebra abstractions and calculations
    \item stb:\ for the easy image loading and decoding
    \item Assimp: As an abstraction for multiple 3D formats that could be
          imported into the project
    \item EnTT:\ for providing the registry in the ECS architecture used
          throughout the project
    \item Bullet3:\ for giving the structures needed to add physics to the scene
    \item CMake:\ for building the project
    \item RenderDoc:\ for debugging the render pipeline
\end{itemize}

The project is subdivided into 7 parts:
\begin{itemize}
    \item Shaders: resources/shaders: all the shaders used in the project
    \item Engine: src/engine: All low-level accesses to OpenGL, resource loading
          and window handling
          \begin{itemize}
              \item The renderer has 2 buffers for the current and previous
                    frame, which are swapped when the renderer needs to send the
                    information coming from the systems to OpenGL. This done in
                    such a way that both updating and rendering are indepedent
                    except from the point where they swap contents. This to make
                    it possible (not currently implemented) to put the updating
                    proces on another thread.
              \item All drawing calls (except some local calls in the renderer
                    itself) are protected with a RAII-guard. This gives the
                    advantage that drawing cannot be done without
                    binding/unbinding as only the guard contains the draw
                    functionality.
          \end{itemize}
    \item Scenes: src/scenes: Contains the scene of the project which bootstraps
          and updates all internal state.
    \item Components: src/components: Contains properties and higher-level
          abstractions used for implementing the game logic.
    \item Prefabs: src/prefabs: Several functions used to create entities in the
          scene and send requests to the engine to load certain resources.
    \item Systems: src/systems: The beating heart of the game part. Takes the
          entities based on their components and does calculations to advance
          the game and send the needed information to the engine. They contain
          the logic of the game.
\end{itemize}

External resources from which a lot of information was gathered on how to do
certain things in OpenGL:
\begin{itemize}
    \item Slides from course
    \item learnopengl.com
    \item docs.gl
    \item open.gl
\end{itemize}

\end{document}
